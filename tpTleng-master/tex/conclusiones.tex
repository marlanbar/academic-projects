\section{Conclusiones}

Los gr\'aficos obtenidos son pr\'acticamente id\'enticos a los provistos
por la c\'atedra. Al comparar los puntos obtenidos con los provistos por la
c\'atedra vemos diferencias de un orden muy peque\~no. Concluimos que se deben
al orden de evaluaci\'on del lenguaje utilizado para interpretar y a la precisi\'on
utilizada en el int\'erprete, en nuestro caso, \texttt{Double}.

A su vez se obtuvo el comportamiento esperado en los archivos de prueba con 
errores, siendo estos rechazados en las etapas correctas con los errores
correctos. 

Con respecto a las herramientas, el implementar el \textit{lexer}, \textit{parser} e int\'erprete
en \textsc{Haskell} nos result\'o muy ventajoso pues las tareas son intrinsecamente
recursivas. Esto se refleja en que ninguno de los archivos supera
las $200$ l\'ineas de c\'odigo y en que el mismo resulta claro, declarativo y
consiso. No podemos decir lo mismo de nuestro primer intento de implementar el
int\'erprete en \textsc{C++}.
