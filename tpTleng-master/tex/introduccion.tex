\section{Introducci\'on}

    El objetivo de este trabajo pr\'actico es implementar un int\'erprete
de un lenguaje de programaci\'on propio, \textit{MyLanga}, que permita graficar funciones.
Para ello creamos 
una gram\'atica basandonos en la especificaci\'on coloquial dada por la 
c\'atedra. 
Luego, utilizando las herramientas \textsc{Alex} y \textsc{Happy} generamos
un \textit{parser} para \textit{MyLanga}. Finalmente utilizamos la estructura generada de
dicho \textit{parser} para interpretarla y generar los puntos de las funciones a graficar.

    El lenguaje \textit{MyLanga} es un lenguaje de \textit{scripting}, imperativo,
con una sintaxis semejante a la de \textsc{Python} o \textsc{C}. Se puede hacer
uso de comentarios multilinea, pero las funcionalidades son acotadas:
no se puede hacer uso de variables globales y las funciones poseen trasparencia
referencial\footnote{Si se las llama dos veces con los mismos argumentos se obtendr\'a
dos veces el mismo resultado.} por ejemplo.
M\'as all\'a de esto, el lenguaje es Turing-completo pues implementa \textit{if},
\textit{while} e, incluso, en el caso de nuestra implementaci\'on, recursi\'on.

    \textsc{Alex} es un analizador l\'exico (de ahora en m\'as \textit{lexer})
basado en expresiones regulares e implementado
en \textsc{Haskell}. Es el an\'alogo a \textsc{Flex} de \textsc{C/C++}.
Permite definir los \textit{tokens}
de la gram\'atica utilizando expresiones regulares. Cada token se denota de
la forma: $regexp$ \{ $code$ \}, donde esto quiere decir: ``si el input es $regexp$
ejecut\'a $code$''.

    \textsc{Happy} es un generador de \textit{parser} implementado en \textsc{Haskell},
an\'alogo a la 
herramienta \textsc{yacc}. Se deben especificar los \textit{tokens} de la gram\'atica,
el m\'odulo del \textit{lexer}, y las pr\'oducciones escritas en notaci\'on
\textsc{BNF}, junto con c\'odigo \textsc{Haskell}, esto es: 
$V_n$ : $(V_n|V_t)^{*}$ \{ $code$ \} 

    El int\'erprete del lenguaje se implement\'o en \textsc{Haskell}, el mismo utiliza el
\textit{parser} generado por \textsc{Happy}, que a su vez utiliza el \textit{lexer} generado
por \textsc{Alex}, y ejecuta el c\'odigo que se le pasa. 


